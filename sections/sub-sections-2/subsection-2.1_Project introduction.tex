\subsection{Giới thiệu đề tài}
\tab Khi phát triển phần mềm hay ứng dụng, bước cần thiết nhất và không thể thiếu
trước khi triển khai sản phẩm vào thực tế là kiểm thử. Có rất nhiều vấn đề ta cần
quan tâm khi tiến hành kiểm thử một ứng dụng, vấn đề bảo mật là một trong số đó. Bảo mật
của một ứng dụng là một vấn đề quan trọng thấy rõ mà nhiều ứng dụng bỏ qua hoặc chưa xử lý, phát hiện
được triệt để các lỗi này. Vì vậy, với tâm lý này làm cho sự xuất hiện của các người dùng tiêu cực và
nhà phát triển tiêu cực ngày càng nhiều và phát triển nhanh chóng, hậu quả gây ra là khôn lường.
Do đó, nhóm sinh viên muốn tạo ra một hệ thống kiểm thử bảo mật trực tuyến, giúp cho việc kiểm thử diễn ra dễ dàng,
nhanh gọn, trực quan hơn. Tên là Owlens - Hệ thống tự động tìm kiếm lỗi bảo mật ứng dụng web dựa trên nền tảng ZAP (Automated web application security scanner based on ZAP)