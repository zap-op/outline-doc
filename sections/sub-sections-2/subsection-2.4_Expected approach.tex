\subsection{Cách tiếp cận dự kiến}
\paragraph{Bản mẫu}
\tab Đang cập nhật\dots

\paragraph{Kiến trúc}
\tab Đang cập nhật\dots

\paragraph{Mô hình dữ liệu}
\tab Đang cập nhật\dots

\paragraph{Các mục tiêu kiểm thử}
\begin{itemize}
    \item Load Testing: đo lường hiệu suất của ứng dụng để xác định tốc độ tương tác.
    \item Reliability Testing: đo lường mức độ xảy ra lỗi của ứng dụng để xác định mức độ lỗi tương tác.
\end{itemize}

\paragraph{So sánh, đánh giá hệ thống với các hệ thống tương tự}
\tab \textbf{Bảng so sánh các tính năng cơ bản của các hệ thống}

\begin{tabularx}{\textwidth}{|>{\hsize=.25\hsize\centering\let\newline
    \\\arraybackslash}X|>{\hsize=.15\hsize\centering\let\newline
    \\\arraybackslash}X|>{\hsize=.15\hsize\centering\let\newline
    \\\arraybackslash}X|>{\hsize=.15\hsize\centering\let\newline
    \\\arraybackslash}X|>{\hsize=.15\hsize\centering\let\newline
    \\\arraybackslash}X|>{\hsize=.15\hsize\centering\let\newline
    \\\arraybackslash}X|}
    \hline
    \textbf{Tính năng}
    & \textbf{...}
    & \textbf{Stack Hawk}
    & \textbf{Detectify}
    & \textbf{Hosted Scan}
    & \textbf{Idyllum Labs}
    \\
    \hline
    Quét với ZAP 
    &
    \checkmark
    &
    \checkmark
    &
    \checkmark
    &
    \checkmark
    &
    \checkmark
    \\
    \hline
    \dots
    &
    &
    &
    &
    &
    \\
    \hline
\end{tabularx}

\vspace{8cm}
\textbf{Bảng ưu điểm và khuyết điểm của các hệ thống}

\begin{tabularx}{\textwidth}{|>{\hsize=0.2\hsize\centering\let\newline
    \\\arraybackslash}X|>{\hsize=0.4\hsize\raggedright\let\newline
    \\\arraybackslash}X|>{\hsize=0.4\hsize\raggedright\let\newline
    \\\arraybackslash}X|}
    \hline
    \thead{Hệ thống}
    & 
    \thead{Ưu điểm}
    & 
    \thead{Khuyết điểm}
    \\
    \hline
    Stack Hawk
    &
    - Tự động quét trên mỗi PR hoặc trong CI/CD.
    \newlinecontenttable
    - Quản lý thông tin kết quả quét, giao diện tường minh, dễ sử dụng.
    \newlinecontenttable
    - Đề xuất liên kết chứa thông tin sửa lỗi tìm được.
    \newlinecontenttable
    - Tích hợp được với những công cụ quản lý khác có sử dụng CI/CD như: Jira Sotfware, Gitlab, Github Action, Azure Pinelines, Jenkins, Travis CI,\dots
    \newlinecontenttable
    - Tài liệu hướng dẫn đầy đủ, chi tiết.
    &
    - Cần cấu hình pineline (.yml, .yaml).
    \newlinecontenttable
    - Cần Docker image đã build xây dựng sẵn.
    \newlinecontenttable
    - Kích hoạt quét bằng command line ở máy local. Cấu hình kích hoạt quét gồm nhiều bước.
    \newlinecontenttable
    - Dùng thử một lần mỗi ngày.
    \\
    \hline
    Detectify
    &
    &
    \\
    \hline
    Hosted Scan
    &
    - Hỗ trợ các loại quét khác như Nmap, OpenVas, SSL/TLS.
    \newlinecontenttable
    - Hẹn thời gian đặt lịch quét.
    \newlinecontenttable
    - Đề xuất liên kết chứa thông tin sửa lỗi tìm được.
    \newlinecontenttable
    - Thông tin scan đầy đủ. Giao diện đơn giản.
    &
    - Kết quả scan dường như là dữ liệu gốc, không được xử lý.
    \newlinecontenttable
    - Dùng thử 10 lần mỗi tháng.
    \\
    \hline
    Idyllum Labs
    &
    - Hỗ trợ các loại quét khác như Nmap, WhatWeb.
    \newlinecontenttable
    - Miễn phí hoàn toàn.
    &
    - Giao diện khó sử dụng.
    \newlinecontenttable
    - Chỉ chấp nhận tên miền gốc (.com,\dots)
    \newlinecontenttable
    - Không theo dõi được quá trình, tiến độ quét.
    \newlinecontenttable
    - Không lựa chọn hay cấu hình được phương thức quét. Tất cả các loại quét có hỗ trợ được quét chung trong một lần.
    \\
    \hline
\end{tabularx}

\paragraph{Đánh giá}
\tab Đang cập nhật\dots

\paragraph{Danh sách các công nghệ, công cụ sử dụng}
\begin{itemize}
    \item Jira Software: một công cụ trong bộ công cụ của ứng dụng quản lý Jira, sử dụng mô hình Kanban để thiết kế và triển khai đồ án.
    \item Confluence: một công cụ trong bộ công cụ của ứng dụng quản lý Jira, dùng để quản lý tài liệu.
    \item GitHub: dịch vụ lưu trữ và kiễm soát phiên bản mã nguồn sử dụng Git.
    \item GitHub Pages: dịch vụ được tạo bởi GitHub, cho phép xuất bản trang web hoặc ứng dụng website bằng cách lưu trữ mã nguồn trong GitHub.
    \item Node.js: môi trường thực thi mã JavaScript bên ngoài trình duyệt.
    \item MongoDB: cơ sở dữ liệu dạng NoSQL.
    \item MongoDB Atlas: cloud database của MongoDB.
    \item OWASP Zed Attack Proxy (ZAP): ứng dụng mã nguồn mở dùng để quét lỗ hổng bảo mật của ứng dụng website.
    \item Express.js: framework dành cho backend Node.js của ứng dụng website.
    \item Server-Send Events - SSEs: một loại thiết hệ thống được xây dựng trên kết nối HTTP. Duy trì kết nối giữa frontend và backend tuy nhiên chỉ có backend được phép gửi dữ liệu lên.
    \item Docker: nền tảng để cung cấp các bước triển khai ứng dụng dễ dàng hơn bằng cách sử dụng các containers.
    \item ReactJS: thư viện JavaScript dùng để xây dựng giao diện website cho người dùng.
    \item Redux: thư viện JavaScript dùng để quản lý và xử lý trạng thái ứng dụng.
    \item Google Identity Services API: dùng để thực hiện định danh bằng Google.
    \item Visual Studio Code: trình biên tập mã được phát triển bởi Microsoft.
\end{itemize}