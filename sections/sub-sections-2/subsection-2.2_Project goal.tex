\subsection{Mục tiêu đề tài}
\paragraph{Thiết kế, xây dựng, kiểm thử và triển khai hệ thống}
\tab Nhóm sinh viên thực hiện mong muốn hoàn thành tối thiểu 80\% - 95\% các chức năng đề ra ban đầu. Các chức năng và mục tiêu chính bao gồm:
\begin{itemize}
    \item Đăng ký / Đăng nhập tài khoản cá nhân để lưu trữ thông tin trong quá trình sử dụng.
    \item Xác thực, cập nhật, bảo mật thông tin tài khoản và lịch sử sử dụng.
    \item Quét ứng dụng web qua URL hoặc IP bằng các loại quét, kịch bản nền tảng ZAP có hỗ trợ.
    \item Hỗ trợ cấu hình các phiên quét ánh xạ với nền tảng ZAP.
    \item Hỗ trợ các loại quét khác như Nmap Port, OpenVas Network Vulnerability và Sslyze TLS/SSL Security.
    \item Quản lý thông tin các URL hoặc IP dùng để quét và các thông tin liên quan.
    \item Giao diện quá trình quét trực quan.
    \item Quản lý các kết quả của các phiên quét.
    \item Giao diện kết quả sau khi quét xong đầy đủ thông tin.
    \item Gửi kết quả trạng thái quét, báo cáo rút gọn qua mail.
    \item Xuất báo cáo với định dạng pdf, xml.
    \item Lịch sử quản lý thông tin các lần quét trước và thông tin kế quả quét.
    \item Các thông tin giới thiệu website của ứng dụng; Trang FAQ; Thông tin liên hệ và hỗ trợ.
    \item Tài liệu đồ án đề tài hoàn thành chi tiết, đầy đủ và bài bản.
\end{itemize}
\paragraph{Tài liệu}
\begin{itemize}
    \item Viết 20 trang Đề cương chi tiết theo mẫu của khoa.
    \item Viết 100 trang khóa luận.
    \item Tài liệu kĩ thuật.
    \begin{itemize}
        \item Hướng dẫn cài đặt công cụ và biên dịch mã nguồn (khoảng 20 trang).
        \item Hướng dẫn triển khai hệ thống (khoảng 30 trang).
        \item Hướng dẫn sử dụng hệ thống (khoảng 20 trang).
    \end{itemize}
\end{itemize}