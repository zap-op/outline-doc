\newpage

\section{NỘI DUNG THỰC HIỆN}

\subsection{Giới thiệu đề tài}
\paragraph{}
Khi phát triển phần mềm hay ứng dụng, bước cần thiết nhất và không thể thiếu 
trước khi triển khai sản phẩm vào thực tế là kiểm thử. Có rất nhiều vấn đề ta cần 
quan tâm khi tiến hành kiểm thử một ứng dụng, vấn đề bảo mật là một trong số đó. Bảo mật 
của một ứng dụng là một vấn đề quan trọng thấy rõ mà nhiều ứng dụng bỏ qua hoặc chưa xử lý, phát hiện
được triệt để các lỗi này. Vì vậy, với tâm lý này làm cho sự xuất hiện của các người dùng tiêu cực và 
nhà phát triển tiêu cực ngày càng nhiều và phát triển nhanh chóng, hậu quả gây ra là khôn lường.
Do đó, chúng tôi muốn tạo ra một hệ thống kiểm thử bảo mật trực tuyến, giúp cho việc kiểm thử diễn ra dễ dàng, 
nhanh gọn, trực quan hơn.

\subsection{Mục tiêu đề tài}
\paragraph{}
Nhóm sinh viên thực hiện mong muốn hoàn thành tối thiểu 80\% - 95\% các chức năng đề ra ban đầu. Các chức năng và mục tiêu chính bao gồm:
\begin{itemize}
    \item Đăng ký / Đăng nhập tài khoản cá nhân để lưu trữ thông tin trong quá trình sử dụng.
    \item Xác thực, cập nhật, bảo mật thông tin tài khoản và lịch sử sử dụng.
    \item Scan webapp thông qua URL; giao diện trực quan, đơn giản.
    \item Xem kết quả scan rút gọn trên web. Xem kết quả scan đầy đủ qua một file report. Xem chi tiết các lần scan trước và các file report đi kèm trong mục lịch sử của tài khoản.
    \item Gửi kết quả về email tài khoản hoặc các email được cài đặt trước.
    \item Các thông tin giới thiệu website, trang FAQ, thông tin liên hệ và hỗ trợ khi cần thiết.
    \item Tận dụng chức năng của chương trình tìm kiếm lỗ hổng bảo mật OWASP ZAP, đưa lên website để dễ dàng tiếp cận người dùng hơn.
    \item Tài liệu đồ án đề tài hoàn thành chi tiết, đầy đủ và bài bản.
\end{itemize} 

\subsection{Phạm vi đề tài}

\subsection{Cách tiếp cận dự kiến}

\subsection{Kết quả dự kiến của đề tài}

\subsection{Kế hoạch thực hiện}